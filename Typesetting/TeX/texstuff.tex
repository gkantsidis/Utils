\usepackage[hyphens]{url}   % This needs to come before hyperref and biblatex

% Bibliography package (avoid printing location and specific date information.
\usepackage[backend=bibtex,abbreviate=true,firstinits=true,isbn=false,url=false,doi=false,eprint=false]{biblatex}
\renewbibmacro{in:}{}
\AtEveryBibitem{%
	\clearfield{day}%
	\clearfield{month}%
	\clearfield{endday}%
	\clearfield{endmonth}%
	\clearlist{location}%
	\clearlist{address}%
	\clearfield{titleaddon}%
	\clearfield{pages}%
	\clearfield{language}%
	\clearlist{editor}%
	\clearfield{series}%
	\clearfield{booktitle}%
}

% Control display of captions
\usepackage[font=footnotesize,labelfont=bf,tableposition=bottom]{caption}
\newenvironment{captiontext}{%
   \begin{center}%
     \begin{minipage}{0.9\linewidth}%
       \renewcommand{\baselinestretch}{0.8}%
         \footnotesize}%
   {\renewcommand{\baselinestretch}{1.0}%
      \end{minipage}%
        \end{center}}
\captionsetup[table]{singlelinecheck=off}

\usepackage{comment}
\usepackage{endnotes}
\usepackage{fancyhdr}
\usepackage{flushend} 					% Balance columns at last page
\usepackage{graphicx}
\usepackage[svgnames]{xcolor}

% Hypertext marks
% http://tug.org/applications/hyperref/ftp/doc/manual.pdf
\usepackage[hyperindex,breaklinks]{hyperref}
% Add capitalization to autoref labels and also use ``Section'' for sub/subsub sections.
\def\sectionautorefname{Section}
\def\subsectionautorefname{Section}
\def\subsubsectionautorefname{Section}
\def\subfigureautorefname{Figure}

\usepackage{breakurl} % Needs to come after hyperref

% http://mirrors.ctan.org/macros/latex/required/psnfss/psnfss2e.pdf
\usepackage{mathptmx} 					% Recommended for SOCC

\usepackage{setspace}
\usepackage{subfigure}
\usepackage{threeparttable}
\usepackage{xspace}

% Do not indent first paragraph of a section
% http://mirror.switch.ch/ftp/mirror/tex/macros/latex/contrib/titlesec/titlesec.pdf
\newcommand{\subparagraph}{}  % Add this dummy variable to avoid error from titlesec
\usepackage{titlesec}
\titlespacing*{\section}{0pt}{1.5ex}{1.5ex}

\usepackage{url}


% This is for units, such as 8GB, 2.4GHz, etc.
\newcommand{\unit}[1]{\,{#1}}

\newlength{\figurewidth}
\setlength{\figurewidth}{0.9\columnwidth}
\newlength{\widefigurewidth}
\setlength{\widefigurewidth}{0.9\textwidth}