% Arsclassica Article
% Structure Specification File
%
% This file has been adapted from:
% http://www.LaTeXTemplates.com
%
% Original author:
% Lorenzo Pantieri (http://www.lorenzopantieri.net) with extensive modifications by:
% Vel (vel@latextemplates.com)
%
% License:
% CC BY-NC-SA 3.0 (http://creativecommons.org/licenses/by-nc-sa/3.0/)
%
%%%%%%%%%%%%%%%%%%%%%%%%%%%%%%%%%%%%%%%%%

%----------------------------------------------------------------------------------------
%	REQUIRED PACKAGES
%----------------------------------------------------------------------------------------

\usepackage[hyphens]{url}   % This needs to come before hyperref and biblatex

\usepackage[
  nochapters, % Turn off chapters since this is an article
  beramono, % Use the Bera Mono font for monospaced text (\texttt)
  eulermath,% Use the Euler font for mathematics
  pdfspacing, % Makes use of pdftex' letter spacing capabilities via the microtype package
  dottedtoc % Dotted lines leading to the page numbers in the table of contents
]{classicthesis} % The layout is based on the Classic Thesis style

\usepackage{arsclassica} % Modifies the Classic Thesis package

\usepackage[T1]{fontenc} % Use 8-bit encoding that has 256 glyphs
\usepackage[utf8]{inputenc} % Required for including letters with accents

% Bibliography package (avoid printing location and specific date information.
\usepackage[
  backend=biber,
  abbreviate=true,
  firstinits=true,
  isbn=false,
  url=false,
  doi=false,
  eprint=false
]{biblatex}
\renewbibmacro{in:}{}
\AtEveryBibitem{%
        \clearfield{day}%
        \clearfield{month}%
        \clearfield{endday}%
        \clearfield{endmonth}%
        \clearlist{location}%
        \clearlist{address}%
        \clearfield{titleaddon}%
        \clearfield{pages}%
        \clearfield{language}%
        \clearlist{editor}%
        \clearfield{series}%
        \clearfield{booktitle}%
}

% Control display of captions
\usepackage[font=footnotesize,labelfont=bf,tableposition=bottom]{caption}
\newenvironment{captiontext}{%
   \begin{center}%
     \begin{minipage}{0.9\linewidth}%
       \renewcommand{\baselinestretch}{0.8}%
         \footnotesize}%
   {\renewcommand{\baselinestretch}{1.0}%
      \end{minipage}%
        \end{center}}
\captionsetup[table]{singlelinecheck=off}

% For including math equations, theorems, symbols, etc
\usepackage{amsmath,amssymb,amsthm}

\usepackage{comment}
\usepackage{endnotes}

% Required for manipulating the whitespace between and within lists
\usepackage{enumitem}

% Balance columns at last page
\usepackage{flushend}

\usepackage{graphicx}
\graphicspath{{Figures/}} % Set the default folder for images

\usepackage{varioref} % More descriptive referencing
\usepackage{breakurl} % Needs to come after hyperref

% http://mirrors.ctan.org/macros/latex/required/psnfss/psnfss2e.pdf
\usepackage{mathptmx}

\usepackage{setspace}
\usepackage{threeparttable}
\usepackage{xspace}

%----------------------------------------------------------------------------------------
%	THEOREM STYLES
%---------------------------------------------------------------------------------------

% Define theorem styles here based on the definition style
% (used for definitions and examples)
\theoremstyle{definition}
\newtheorem{definition}{Definition}

% Define theorem styles here based on the plain style
% (used for theorems, lemmas, propositions)
\theoremstyle{plain}
\newtheorem{theorem}{Theorem}

% Define theorem styles here based on the remark style (used for remarks and notes)
\theoremstyle{remark}

% This is for units, such as 8GB, 2.4GHz, etc.
\newcommand{\unit}[1]{\,{#1}}

\newlength{\figurewidth}
\setlength{\figurewidth}{0.9\columnwidth}
\newlength{\widefigurewidth}
\setlength{\widefigurewidth}{0.9\textwidth}

%%% Local Variables:
%%% mode: latex
%%% TeX-master: t
%%% End:
