%%%%%%%%%%%%%%%%%%%%%%%%%%%%%%%%%%%%%%%%%
% Arsclassica Article
% Structure Specification File
%
% This file has been downloaded from:
% http://www.LaTeXTemplates.com
%
% Original author:
% Lorenzo Pantieri (http://www.lorenzopantieri.net) with extensive modifications by:
% Vel (vel@latextemplates.com)
%
% License:
% CC BY-NC-SA 3.0 (http://creativecommons.org/licenses/by-nc-sa/3.0/)
%
%%%%%%%%%%%%%%%%%%%%%%%%%%%%%%%%%%%%%%%%%

%----------------------------------------------------------------------------------------
%	REQUIRED PACKAGES
%----------------------------------------------------------------------------------------


\usepackage[
nochapters, % Turn off chapters since this is an article
beramono, % Use the Bera Mono font for monospaced text (\texttt)
eulermath,% Use the Euler font for mathematics
pdfspacing, % Makes use of pdftex’ letter spacing capabilities via the microtype package
dottedtoc % Dotted lines leading to the page numbers in the table of contents
]{classicthesis} % The layout is based on the Classic Thesis style

\usepackage{arsclassica} % Modifies the Classic Thesis package

\usepackage[T1]{fontenc} % Use 8-bit encoding that has 256 glyphs

\usepackage[utf8]{inputenc} % Required for including letters with accents

% Bibliography package (avoid printing location and specific date information.
\usepackage[
  backend=biber,
  abbreviate=true,
  giveninits=true,
  isbn=false,
  url=false,
  doi=false,
  eprint=false
]{biblatex}
\renewbibmacro{in:}{}
\AtEveryBibitem{%
        \clearfield{day}%
        \clearfield{month}%
        \clearfield{endday}%
        \clearfield{endmonth}%
        \clearlist{location}%
        \clearlist{address}%
        \clearfield{titleaddon}%
        \clearfield{pages}%
        \clearfield{language}%
        \clearlist{editor}%
        \clearfield{series}%
        \clearfield{booktitle}%
}

%
% Control display of captions
% You could change the names below to Fig., Tab. for shorter versions.
\usepackage[font=footnotesize,labelfont=bf,tableposition=bottom,figurename=Figure,tablename=Table]{caption}
\newenvironment{captiontext}{%
   \begin{center}%
     \begin{minipage}{0.9\linewidth}%
       \renewcommand{\baselinestretch}{0.8}%
         \footnotesize}%
   {\renewcommand{\baselinestretch}{1.0}%
      \end{minipage}%
        \end{center}}
\captionsetup[table]{singlelinecheck=off}

% For including math equations, theorems, symbols, etc
\usepackage{amsmath,amssymb,amsthm}

\usepackage{comment}
\usepackage{endnotes}
\usepackage{todonotes}
\makeatletter
\renewcommand*\makeenmark{\hbox{\textsuperscript{\@Alph{\theenmark}}}}
\makeatother

\usepackage{graphicx} % Required for including images
\graphicspath{{Figures/}  } % Set the default folder for images, observe the double curly brackets.

\usepackage{enumitem} % Required for manipulating the whitespace between and within lists
\setlist{noitemsep} % Remove spacing between bullet/numbered list elements

\usepackage{lipsum} % Used for inserting dummy 'Lorem ipsum' text into the template

\usepackage{subfig} % Required for creating figures with multiple parts (subfigures)

\usepackage{varioref} % More descriptive referencing

% Control list of authors and their affiliations
\usepackage[auth-sc]{authblk}
\renewcommand\Affilfont{\itshape\small}

\usepackage{xargs}
\usepackage{xspace}

% For units
\usepackage[unit-optional-argument,use-xspace,binary-units]{siunitx}
% For standard phrases such as eg., i.e., a priori, etc.
\usepackage[all]{foreign}

\usepackage[boxed,noline]{algorithm2e}
\usepackage{tikz}
\usetikzlibrary{trees}

\usepackage{flushend}

%----------------------------------------------------------------------------------------
%	THEOREM STYLES
%---------------------------------------------------------------------------------------

\theoremstyle{definition} % Define theorem styles here based on the definition style (used for definitions and examples)
\newtheorem{definition}{Definition}

\theoremstyle{plain} % Define theorem styles here based on the plain style (used for theorems, lemmas, propositions)
\newtheorem{theorem}{Theorem}
\newtheorem{axiom}[theorem]{Axiom}
\newtheorem{case}[theorem]{Case}
\newtheorem{conclusion}[theorem]{Conclusion}
\newtheorem{condition}[theorem]{Condition}
\newtheorem{conjecture}[theorem]{Conjecture}
\newtheorem{corollary}[theorem]{Corollary}
\newtheorem{criterion}[theorem]{Criterion}
\newtheorem{example}[theorem]{Example}
\newtheorem{exercise}[theorem]{Exercise}
\newtheorem{lemma}[theorem]{Lemma}
\newtheorem{notation}[theorem]{Notation}
\newtheorem{problem}[theorem]{Problem}
\newtheorem{proposition}[theorem]{Proposition}
\newtheorem{remark}[theorem]{Remark}
\newtheorem{solution}[theorem]{Solution}
\newtheorem{summary}[theorem]{Summary}

\theoremstyle{remark} % Define theorem styles here based on the remark style (used for remarks and notes)
\newtheorem{acknowledgement}[theorem]{Acknowledgement}

%----------------------------------------------------------------------------------------
%	HYPERLINKS
%---------------------------------------------------------------------------------------

\hypersetup{
%draft, % Uncomment to remove all links (useful for printing in black and white)
colorlinks=true, breaklinks=true, bookmarks=true,bookmarksnumbered,
urlcolor=webbrown, linkcolor=RoyalBlue, citecolor=webgreen, % Link colors
pdftitle={\ArticleTitle}, % PDF title
pdfauthor={\textcopyright}, % PDF Author
pdfsubject={}, % PDF Subject
pdfkeywords={}, % PDF Keywords
pdfcreator={pdfLaTeX}, % PDF Creator
pdfproducer={LaTeX with hyperref and ClassicThesis} % PDF producer
}

\def\sectionautorefname{Sec.}
\def\subsectionautorefname{Sec.}
\def\subsubsectionautorefname{Sec.}
\def\figureautorefname{Fig.}
\def\subfigureautorefname{Fig.}
\def\algorithmautorefname{Algorithm}

\newcommand{\vautoref}[1]{\sectionautorefname~\vref{#1}\xspace}

% define a macro \Autoref to allow multiple references to be passed to \autoref
\makeatletter
\newcommand\Autoref[1]{\@first@ref#1,@}
\def\@throw@dot#1.#2@{#1}% discard everything after the dot
\def\@set@refname#1{%    % set \@refname to autoefname+s using \getrefbykeydefault
    \edef\@tmp{\getrefbykeydefault{#1}{anchor}{}}%
    \xdef\@tmp{\expandafter\@throw@dot\@tmp.@}%
    \ltx@IfUndefined{\@tmp autorefnameplural}%
         {\def\@refname{\@nameuse{\@tmp autorefname}s}}%
         {\def\@refname{\@nameuse{\@tmp autorefnameplural}}}%
}
\def\@first@ref#1,#2{%
  \ifx#2@\autoref{#1}\let\@nextref\@gobble% only one ref, revert to normal \autoref
  \else%
    \@set@refname{#1}%  set \@refname to autoref name
    \@refname~\ref{#1}% add autoefname and first reference
    \let\@nextref\@next@ref% push processing to \@next@ref
  \fi%
  \@nextref#2%
}
\def\@next@ref#1,#2{%
   \ifx#2@ and~\ref{#1}\let\@nextref\@gobble% at end: print and+\ref and stop
   \else, \ref{#1}% print  ,+\ref and continue
   \fi%
   \@nextref#2%
}
\makeatother

